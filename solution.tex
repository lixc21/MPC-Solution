\documentclass[11pt,a4paper]{report}
\usepackage[hmargin=1.25in,vmargin=1in]{geometry}
\usepackage{amsthm,cite,url,amsmath,amssymb,bm}
\usepackage{algorithm,graphicx,color,mathtools}
\usepackage{physics,enumitem,thmtools}
\usepackage{hyperref,tikz}


\title{The Solution of Model Predictive Control: Theory, Computation, and Design \cite{rawlings2017model}}
\author{lixc21}
\date{\today}



\newtheorem*{remark}{Remark}
\theoremstyle{definition}\newtheorem{exercise}{Exercise}[chapter]
\declaretheoremstyle[
  headfont=\color{blue}\normalfont\bfseries,
  bodyfont=\color{blue}\normalfont,
]{colored}
\declaretheorem[
  style=colored,
  name=Answer,
]{answer}



\begin{document}
\maketitle

\chapter{Getting Started with Model Predictive Control}
\section{Brief Review}
In this section, we just consider state space linear time invariant system with zero steady state.

\paragraph{Lemma 1.3} (LQR convergence). For $(A,B)$ controllable, the infinite LQR gives a convergent closed-loop system.
\begin{proof}
Because $(A,B)$ is controllable, there exists a sequence of $n$ inputs that transfers the state to zero. When $k>n$, we let $u=0$, then the objective function $V(x,u)=\sum_{k=0}^\infty x_k^\top Qx_k+u^\top Ru$ is finite, which implies the optimization problem is feasible. On the other hand, the solution is unique since $R>0$ and the objective function is strict convex with $u$.

So the solution of the LQR problem exists and is unique. This implies to that the objective function is non-increasing with time, and we have $x\to 0$, $u\to 0$ as $k\to 0$.
\end{proof}
\begin{remark}
The optimal solution can be calculate from Riccati equation, which is from backward dynamic programming similar to Kalman filter.
\begin{equation}\notag
\begin{aligned}
K&=-(B^\top PB+R)^{-1} B^\top PA\\
P&=Q+A^\top PA-A^\top PB(B^\top PB+R)^{-1}B^\top PA
\end{aligned}
\end{equation}
\end{remark}


\section{The Solution of Exercises}
\begin{exercise} State space form for chemical reaction model.\\
Consider the following chemical reaction kinetics for a two-step series reaction
\begin{equation}
    A\xrightarrow{k_1} B\qquad B\xrightarrow{k_2} C
\end{equation}
We wish to follow the reaction in a constant volume, well-mixed, batch reactor. As taught in the undergraduate chemical engineering curriculum, we proceed by writing material balances for the three species giving
\begin{equation}
    \dv{c_A}{t}=-r_1\qquad \dv{c_B}{t}=r_1-r_2\qquad \dv{c_C}{t}=r_2
\end{equation}
in which $c_j$ is the concentration of species $j$, and $r_1$ and $r_2$ are the rates $\rm (mol/(time\cdot vol))$ at which the two reactions occur. We then assume some rate law for the reaction kinetics, such as
\begin{equation}
    r_1=k_1 c_A\qquad r_2=k_2 c_B
\end{equation}
We substitute the rate laws into the material balances and specify the starting concentrations to produce three differentia equations for the three species concentrations. 

\begin{enumerate}[label=(\alph*)]
    \item write the linear state space model for the deterministic series chemical reaction model. Assume we can measure the component A concentration. What are $x$, $y$, $A$, $B$, $C$, and $D$ for this model?
    \item Simulate this model with initial conditions and parameters given by $$c_{A0}=1\quad c_{B0}=c_{C0}=0\quad k_1=2\quad k_2=1$$
\end{enumerate}
\end{exercise}

\begin{answer}
\begin{enumerate}[label=(\alph*)]
    \item the linear state space model is
    \begin{equation}
        \dv{x}{t} = 
        \begin{bmatrix}
            -k_1 &0 &0 \\
            k_1 &-k_2 &0 \\
            0 &k_2 &0 \\
        \end{bmatrix} x = A x
    \end{equation}
    where $x=[c_A,\ c_B,\ c_C]^\top$. $B$ does not exist because there is no system input variables. $C=[1,\ 0,\ 0]^\top$, $D=0$, $y=Cx$.
    
    \item the simulation result is shown as Fig.\ref{fig:exer1}. The code used in all of the exercise can be found in github \url{https://github.com/lixc21/MPC-Solution}.
\end{enumerate}
\begin{figure}[htbp]
    \centering
    \includegraphics[width=\linewidth/2]{./code_ch1/exer1.pdf}
    \caption{system simulation}
    \label{fig:exer1}
\end{figure}
\end{answer}

\begin{exercise} Distributed systems and time delay.\\
We assume familiarity with the transfer function of a time delay from an undergraduate systems course
\begin{equation}
    \bar{y}(s)=e^{-\theta s}\bar{u}(s)
\end{equation}
Let's see the connection between the delay and the distributed systems, which give rise to it. A simple physical example of a time delay caused by transport in a flowing system. Consider plug flow in a tube depicted in Fig.\ref{fig:exer2}.

\begin{figure}
\centering
\tikzset{every picture/.style={line width=0.75pt}} %set default line width to 0.75pt        
\begin{tikzpicture}[x=0.75pt,y=0.75pt,yscale=-1,xscale=1]
%uncomment if require: \path (0,185); %set diagram left start at 0, and has height of 185

%Shape: Ellipse [id:dp19295851654284357] 
\draw   (272,42) .. controls (283.05,42) and (292,57.67) .. (292,77) .. controls (292,96.33) and (283.05,112) .. (272,112) .. controls (260.95,112) and (252,96.33) .. (252,77) .. controls (252,57.67) and (260.95,42) .. (272,42) -- cycle ;
%Straight Lines [id:da7486939098222503] 
\draw    (272,42) -- (399.44,42) ;
%Straight Lines [id:da3129886311233703] 
\draw    (272,112) -- (399.44,112) ;
%Shape: Arc [id:dp1273018290846848] 
\draw  [draw opacity=0] (399.44,42) .. controls (399.44,42) and (399.44,42) .. (399.44,42) .. controls (399.44,42) and (399.44,42) .. (399.44,42) .. controls (410.49,42) and (419.44,57.67) .. (419.44,77) .. controls (419.44,96.33) and (410.49,112) .. (399.44,112) -- (399.44,77) -- cycle ; \draw   (399.44,42) .. controls (399.44,42) and (399.44,42) .. (399.44,42) .. controls (399.44,42) and (399.44,42) .. (399.44,42) .. controls (410.49,42) and (419.44,57.67) .. (419.44,77) .. controls (419.44,96.33) and (410.49,112) .. (399.44,112) ;  
%Straight Lines [id:da8609030882922339] 
\draw    (123,76.67) -- (237.44,77.32) ;
\draw [shift={(239.44,77.33)}, rotate = 180.33] [color={rgb, 255:red, 0; green, 0; blue, 0 }  ][line width=0.75]    (10.93,-3.29) .. controls (6.95,-1.4) and (3.31,-0.3) .. (0,0) .. controls (3.31,0.3) and (6.95,1.4) .. (10.93,3.29)   ;
%Straight Lines [id:da8626778179325296] 
\draw    (437.44,77.33) -- (536.11,77.33) ;
\draw [shift={(538.11,77.33)}, rotate = 180] [color={rgb, 255:red, 0; green, 0; blue, 0 }  ][line width=0.75]    (10.93,-3.29) .. controls (6.95,-1.4) and (3.31,-0.3) .. (0,0) .. controls (3.31,0.3) and (6.95,1.4) .. (10.93,3.29)   ;

% Text Node
\draw (124,41.4) node [anchor=north west][inner sep=0.75pt]    {$c_{j} \ ( 0,t) =u( t)$};
% Text Node
\draw (124,84.4) node [anchor=north west][inner sep=0.75pt]    {$v$};
% Text Node
\draw (252,118.4) node [anchor=north west][inner sep=0.75pt]    {$z=0$};
% Text Node
\draw (390,123.4) node [anchor=north west][inner sep=0.75pt]    {$z=L$};
% Text Node
\draw (439,41.4) node [anchor=north west][inner sep=0.75pt]    {$c_{j} \ ( 0,t) =y( t)$};
\end{tikzpicture}
\caption{Plug-flow reactor}
\label{fig:exer2}
\end{figure}

\begin{enumerate}[label=(\alph*)]
    \item Write down the equation of change for moles of component $j$ for an arbitrary volume element and show that
    \begin{equation}
        \pdv{c_j}{t}=-\nabla \cdot (c_j v_j)+R_j
    \end{equation}
    in which $c_j$ is the molar concentration of component $j$, $v_j$ is the velocity of component $j$, and $R_j$ is the production rate of component $j$ due to chemical reaction.

    Plug flow means the fluid velocity of all components os purely in the $z$ direction, and os independent of $r$ and $\theta$ and, we assume here, z
    \begin{equation}
        v_j=v\delta_z
    \end{equation}
    \item Assuming plug flow and neglecting chemical reaction in the tube, show that the equation of change reduces to 
    \begin{equation}
        \pdv{c_j}{t}=-v\pdv{c_j}{z}
    \end{equation}
    This equation is known as a hyperbolic, first-order partial differential equation.
    \begin{alignat}{3}
        &c_j(z,t)=u(t) &&0=z &&t\ge 0  \\
        &c_j(z,t)=c_{j0}(t) \quad &&0\le z\le L\quad &&t=0 
    \end{alignat}
    In other words, we are using the feed concentration as the manipulated bariable, 

\end{enumerate}


\end{exercise}

















\chapter{Appendix A. Mathematical Background}
Since the mathematical background is basic, we jump to the exercises section.
\section{The solution of Exercises}
\begin{exercise} Norms in $\mathbb{R}^n$\\
Show that the following three functions are all norms in $\mathbb{R}^n$

\begin{equation}\notag
    \|x\|_2 
\end{equation}

\end{exercise}

\bibliographystyle{unsrt}
\bibliography{mybib}

\end{document}






















